\documentclass[11pt,a4paper]{moderncv}    
\moderncvstyle{banking}
\moderncvcolor{black}                 
\definecolor{color2}{rgb}{0.25, 0.25, 0.25}
\usepackage[T1]{fontenc}  
\usepackage{microtype}      
\usepackage{multicol}                       
\usepackage{lastpage}                    
\usepackage[scale=0.86]{geometry}
\usepackage[absolute,overlay]{textpos} 
\usepackage{mathptmx}
\usepackage{xcolor}
\usepackage{enumitem}

\setlist[enumerate,itemize]{itemsep=0pt, leftmargin=15pt}
\renewcommand{\sectionfont}{\Large\bfseries\scshape}
\renewcommand{\subsectionrule}{}
\renewcommand{\footrulewidth}{0.5pt}
\renewcommand{\footrule}{\hbox to\headwidth{\color{gray!40}\leaders\hrule height \footrulewidth\hfill}}
\rfoot{\thepage~$\big\vert$ \color{gray}\textls[200]{Page}}
%%%%%%%%%%%%%%%%%%%%%%%%%%%%%%%%%%%%%%%%%%%%%%%%%%%%%%%%%%%%%%%%%%%%%%%%%%%%%%%%
%\title{Curriculum vitae}
\name{Md. Jubayer}{Hossain}
\address{Azimpur, 7/A, Dhaka}{Bangladesh}
\email{b150605021@mib.jnu.ac.bd} 
\homepage{https://jhossain.netlify.app/} 
\phone[mobile]{+8801771083855}             
%\social[linkedin]{jubayer28}

\begin{document}

\hskip-2.5cm\makecvtitle
%\begin{textblock}{0}(12.3,1)
%\includegraphics[scale=0.4]{photo1}
%\end{textblock}
\section{Academic Credentials}
\begin{itemize}
\item \textbf{Master of Science} \hfill [On Going]\\
\textit{Department of Microbiology, Jagannath University, Dhaka}
\item \textbf{Bachelor of Science} \hfill [Jan 2015--Jan 2019]\\
\textit{Department of Microbiology, Jagannath University, Dhaka\\ GPA: 
3.15/4.00}
\end{itemize}

\section{Work Experience}
\begin{itemize}
	\item \textbf{Lead Organizer and Instructor}, 
	\textcolor{blue}{\href{https://scicomforbio.github.io/}{Scientific Computing for Biologists}} \hfil[Nov 2020 - Present] \\
	Introduction to Scientific Computing for Biologists is an introductory 
	course organized by Center for Health Innovation, Research, Action and Learning--Bangladesh (CHIRAL Bangladesh). My team members and I design this course to help  undergrad students interested in genomics and
	public health research. Here I taught the basics of Python, R programming, 
	Biostatistics, and Bioconductor.
	\item \textbf{Lead Organizer and Instructor}, 
	\textcolor{blue}{\href{https://scicomforbio.github.io/}{Health Data Analytics}} \hfil[July 2021 - August 2021] \\
	Python for Health Data Analytics is an introductory 
	course organized by Center for Health Innovation, Research, Action and Learning--Bangladesh(CHIRAL Bangladesh).
	\item \textbf{Instructor}, 
	\textcolor{blue}{\href{}{Whitehope}}
	\hfill [July 2016 - June 2020]\\ 
	I provided comprehensive tutoring in Physics and Chemistry to the 
	students of 10th, 11th, and 12th grade.
	
	\item \textbf{Team Member}, 
	\textcolor{blue}{\href{}{Bio-Bio-1}}
	\hfill [June 2017 - Jan 2019]\\ 
	I was an active participant in their weekly study circle and learned a lot 
	from them; especially, I learned bioinformatics techniques with the 
	help of Python.
	
\end{itemize}

\section{Research}
\begin{enumerate}
	\item Syeda Tasneem Towhid, Ph.D, \textbf{Md. Jubayer Hossain}, Sumona Akter, Md. 
	Atik Shariar Sammo (2020). Perception of Students on Antibiotic Resistance 
	and Prevention: An internet-based community case study from 
	Dhaka,Bangladesh. \hfill [Under Review]
	\item Mohammad Ariful Islam, Ph.D, \textbf{Md. Jubayer Hossain}, Sumaiya Akter Mukta, Rubaiya Gulshan, Md. Sharif Miah (2021). Knowledge and Attitudes of Thalassemia among Public University Students in Bangladesh\hfill [In Preparation]
	\item Mohammad Ariful Islam, Ph.D, \textbf{Md. Jubayer Hossain},Mousumi Karmakar, Sumaiya Akter Mukta, Rubaiya Gulshan, Md. Sharif Miah (2021). Quality of Life among Bangladeshi Patients with Thalassemia using the SF-36 Questionnaire\hfill [In Preparation]
	\item Md. Kamrujjaman, Ph.D, Alamgir Hossain Tanjim, \textbf{Md. Jubayer 
	Hossain} (2021). A survey on the general concept of diabetes among students 
	of different universities in Bangladesh.\hfill [In Preparation]
	\item\textbf{ Md. Jubayer Hossain}, Sumona Akter, Md. Atik Shariar Sammo, 
	Shahid Hasan Munzur (2021). Self Management And Knowledge of Diabetic 
	Patients in Bangladesh and the Prevalence rate of Diabetes.\hfill [In 
	Preparation]
\end{enumerate}

%\section{Projects}
%\begin{itemize}	
%	\item\textbf{Disease Detection System} 
%	\hfill [\href{https://github.com/hdrobd/DDS}{Github}]\\
%	The aim of this Disease Detection System(DDS) to detect some disease by 
%	using machine learning techniques.
%\end{itemize}

%
%\section{Academic Projects}
%\begin{itemize}	
%	\item 
%	\textbf{\href{https://github.com/hossainlab/ARGs}{Bioinformatics
%	 Approach for Profiling Antibiotic Resistance genes from Metagenomic DNA 
%	Sequence}} \\  (Advisor: \textit{Dr. Shuvro 
%	Nandi}) \hfill [February, 2020]
%	\item\textbf{\href{https://github.com/hossainlab/ArtofGenomeAnalysis}{State 
%	the Art of Microbial Genome Analysis}} \\  (Advisor: 
%	\textit{Dr. Shamima Begum}) \hfill [October, 2020]
%\end{itemize}


% Skills Section
\section{KEY SKILLS}
\begin{multicols}{2}
	\begin{itemize}
		\item \textbf{Operating Systems:} Windows, Linux
		\item \textbf{Version Control:} Git, Github
	    \item \textbf{Programming Languages:} Python, R, Julia, SQL
		\item \textbf{Statistical Software Packages:} SPSS, MS Excel
		
	\end{itemize}
	\columnbreak 
	\begin{itemize}
		\item \textbf{Survey Tools:} KoboToolBox, Google Forms
		\item \textbf{Typesetting and Illustrations:} \LaTeX, MS Word, MS PowerPoint
		\item \textbf{BioStack:} Biopython, Scikit-bio, BioPandas, Biotte, RDkit
	\end{itemize}
\end{multicols} 


\section{Affiliations}
\begin{itemize}
	\item \textbf{Founder}, 
	\textcolor{blue}{\href{https://hdrobd.org/}{Center for Health Innovation, Research, Action and Learning -- Bangladesh (CHIRAL Bangladesh)}}
	\hfill [June 2020 - Present]\\ 
	The main focus of this group is to solve healthcare problems with the help 
	of data in Bangladesh. The primary responsibilities include designing 
	courses, data analysis, and team management.
	\item \textbf{Co-Founder}, 
	\textcolor{blue}{\href{}{Apon}}
	\hfill [2016 - Present]\\
	Apon is a non-profit blood donation platform. Primarily Apon is working for 
	Thalassemia patients in Bangladesh.
	\item \textbf{Founder}, 
	\textcolor{blue}{\href{}{Pounopunik}}
	\hfill [Nov 2016 - Present]\\
	The Pounopunik is a science organization at Sreepur. This organization aims 
	to arrange a science olympiad and other scientific events for primary, 
	secondary, and higher secondary students. 

\end{itemize}


\section{Linguistic Proficiency}
\begin{itemize}  	
	\item English (Fluent Working Proficiency)
	\item Bangla (Native Language)
\end{itemize}



\section{Extra Curricular Activities}
\begin{itemize}
	\item \textbf{Founding Member}, 
	\textcolor{blue}{\href{}{Kanthokanon}}
	\hfill [Nov 2016 - Present]\\ 
	Kanthokanon is a poetry recitation organization at Sreepur. Here I am an 
	active member. I learned Bangla pronunciation and poem recitation. I 
	performed in 3 programs and won 2 awards from "Victory Day Poem Recitation 
	Competition-2014."
	
	\item \textbf{Participant}, 
	\textcolor{blue}{\href{}{Recitation, Presentation and News Reading 
	Workshop}}
	\hfill [March 2013]\\ 
	Kanthobithy Magura arranged the workshop in 2013. Kanthobithy Magura is the 
	first poetry recitation organization in Magura. 
	
	
\end{itemize}

\section{References}
\begin{multicols}{2}
\textbf{Syeda Tasneem Towhid, Ph.D}\\ 
Assistant Professor, \\ 
Department of Microbiology \\ 
Jagannath University, Dhaka \\ 
Email -- \textcolor{blue}{\url{towhidst@mib.jnu.ac.bd}} 

\columnbreak 
\textbf{Mohammad Ariful Islam, Ph.D}\\ 
Associate Professor,\\ 
Department of Microbiology \\ 
Jagannath University, Dhaka \\ 
Email -- \textcolor{blue}{\url{ariful@mib.jnu.ac.bd}}
\end{multicols} 
\end{document}


