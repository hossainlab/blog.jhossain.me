% Source: http://tex.stackexchange.com/a/150903/23931
\documentclass{article}
\usepackage[letterpaper,margin=0.5in]{geometry}
\usepackage{xcolor}
\usepackage{fancyhdr}
%\usepackage{tgschola} % or any other font package you like
\usepackage{mathptmx}
\usepackage[scale=0.90,lf]{FiraMono}
%\pagestyle{fancy}
%\fancyhf{}
%\fancyhead[C]{
%  \footnotesize\sffamily
%  \yourname\quad
%  Personal Website: \textcolor{blue}{\itshape\yourweb}\quad
%  Email: \textcolor{blue}{\youremail}}
\newcommand{\soptitle}{Statement of Purpose}
%\newcommand{\yourname}{Md. Jubayer Hossain}
%\newcommand{\youremail}{b150605021@gmail.com}
%\newcommand{\yourweb}{https://jhossain.netlify.app/}

\newcommand{\statement}[1]{\par\medskip
  \underline{\textcolor{blue}{\textbf{#1}}}\space
}

\usepackage[
  colorlinks,
  breaklinks,
  pdftitle={\soptitle},
  %pdfauthor={\yourname},
  unicode
]{hyperref}

\begin{document}
\begin{center}\LARGE\soptitle\\
%\large \yourname,\ Applicant for MPH Program in Biostatistics, Fall---2022
\end{center}
\hrule
\vspace{1pt}
\hrule height 1pt
\bigskip

I am a student of Microbiology from Bangladesh, a low-middle income country in South Asia. I have made it my career goal to address the public health crises in the developing world. My application for Master of Public Health in Biostatistics comes as the first step towards achieving that goal. I am applying for the Master of Public Health Program is offered by the School of Public Health at the University of Florida. I have wanted to do meaningful translational research since the beginning for undergraduate studies. I became interested in Healthcare when the COVID-19 pandemic in Bangladesh in 2019 has been regarded as the most challenging situation in post-modern history. As an undergraduate student at the country’s public university, it made me realize the appalling level of inadequate healthcare management in Bangladesh. I became determined to change healthcare management and include data-driven decisions in a desperately needed healthcare reform. While thinking about our country’s healthcare problems, I founded an organization named ”Center for Health Innovation, Research, Action, and Learning - Bangladesh (nickname; CHIRAL Bangladesh) to integrate modern health data science and data-driven decisions into policymaking and action. In my current position, I am working with internationally acclaimed researchers to implement efficient healthcare management. As plenty of data are being generated in public health and biomedical research, biostatistics is applied in the growing number of studies related to human health to draw valid conclusions. Therefore, a career in biostatistics perfectly suits my interest in using statistical methods and eagerness to advance public health. In pursuing my Master’s degree, I wish to advance healthcare management by learning and applying advanced statistics and data science to improve healthcare efficiency. In addition, I want to investigate specialized technologies and strategies for use in Healthcare. I believe that the vast resources at the University of Florida will complement my academic training, research experience, and leadership in pursuing community service healthcare projects in my journey to become an expert in healthcare management, policymaking strategies. 


Currently, I am working on several public health projects from the perspective of Bangladesh, including Antibiotic Resistance and Prevention in collaboration with Dr. Syeda Tasneem Towhid, Post-doctoral researcher at Ottawa University, Canada, and director of the Center for Health Innovation, Research, Action, and Learning - Bangladesh. I have the privilege of conducting my first study entitled ”Perception of Students on Antibiotic Resistance and Prevention: An Online, Community-Based Case Study from Dhaka, Bangladesh.” The study is under review for publication. During the investigation, I analyzed the current perception and the knowledge level about antibiotic usage and misuse of Bangladeshi Students from different backgrounds. Thalassemia: Thalassemia is an enormous public health problem in Bangladesh. 7\% of our population are thalassemia carriers. Every year, 7000 new babies are born with thalassemia; my role at the Center for Health Innovation, Research, Action, and Learning -Bangladesh (CHIRAL Bangladesh) also allowed me to collaborate research with Bangladesh Thalassemia Foundation. The plans for the prevention and awareness of Thalassemia among Bangladeshi adults are conducting two studies under the supervision of Dr. Mohammad Ariful Islam: ”Knowledge and Attitudes of Thalassemia among Public University Students in Bangladesh,” and the other is ”Quality of Life among Bangladeshi Patients with Thalassemia using the SF-36 Questionnaire.” My contribution to these studies is study design, questionnaire development, and data analysis. Diabetes: In Bangladesh, 8.4 million adults were being with diabetes in 2019 and predicted to almost double (15.0 million) by 2045. It is also estimated that another 3.8 million people had prediabetes in Bangladesh in 2019. I am working on two papers on diabetes; one is ”General Concept of Diabetes among Villagers of Different Districts in Bangladesh: A Cross-sectional Study” under Dr. Md. Kamrujjaman, an Associate Professor of Mathematics at Dhaka University, and another one is ”Self-Management And Knowledge of Diabetic Patients in Bangladesh and the Prevalence rate of diabetes.” COVID-19: COVID-19 affects people worldwide; the world moves physical class to the virtual classroom, known as distance learning. We have launched a project named ”Distance Learning” at the Center for Health Innovation, Research, Action, and Learning - Bangladesh to explore the effects of distance learning on the university, school, and college students in Bangladesh. We are also working on the quality of life of Bangladeshi adults during the COVID19 pandemic—this project under Sabia Sultana, Undergrad Research Coordinator of Center for Health Innovation, 1 Research, Action and Learning - Bangladesh, Ph.D. Candidate at Simon Fraser University, Canada, and Assistant Professor of Microbiology at Jagannath University in Dhaka. Besides research, I am the lead organizer and instructor of some courses at the Center for Health Innovation, Research, Action, and Learning - Bangladesh. Introduction to Scientific Computing for Biologists: This is an introductory course for undergrad students with a biology background interested in Bioinformatics and Genomic Data Science. I taught the fundamentals of Python, R Programming, Biostatistics, Bioconductor, and LaTeX. Python for Health Data Analytics: The course covered the basics of Python Programming Language, and the data analysis workflow focuses on health data. I taught NumPy, data management with Pandas, data visualization with matplotlib and seaborn, and linear regression and regression model building and evaluating with scikit-learn. Health Research ToolBox–A Step by Step Guide for Beginners. The course on health research methodology and necessary research tools including research methods in health science, KoboToolBox for data collection, fundamentals of the statistics for research, analysis, and imprecation data with SPSS, and academic writing with LaTeX. My leadership experience has made me face the challenges and complexities of Healthcare Management in Bangladesh, a developing country where healthcare decisions are made without proper data-driven techniques. My vision is to apply the latest science and technologies to ensure a safe, economic, and perfect environment-friendly healthcare system. Under the guidance of the premier faculty at the University of Florida, I aspire to become an expert in Healthcare management. I would be looking forward to a long and fruitful association with the University of Florida.

I hope you can see how driven I am by the range of activities I am involved in despite the resource-limited setting in my home country, even without a formal degree in public health. Formal education in the relevant field from an international institute would enable me to make meaningful changes that help shape public health management to efficiently prevent emerging threats to public health. 

\end{document}